\documentclass{beamer}
\usepackage{luatextra}
\usepackage{polyglossia}
\usepackage{ulem}
\usepackage{framed}
\usepackage{color}
\usepackage{geometry}
\usepackage{amsmath}
\usepackage{unicode-math}
\usepackage{hyperref}

\usepackage{ifluatex}
\ifluatex
  \usepackage{pdftexcmds}
  \makeatletter
  \let\pdfstrcmp\pdf@strcmp
  \let\pdffilemoddate\pdf@filemoddate
  \makeatother
\fi
\usepackage{svg}

%%\setmathfont{xits-math.otf}

\setmainlanguage{french}
\setmainfont{Latin Modern Roman}

%%\geometry{margin={1in,1in}}

\newcommand\image[2]{
\directlua{
local image = img.scan({filename = "#1"})

image.height = image.height * #2
image.width  = image.width  * #2

node.write(img.node(image))
}
}

\usetheme{Madrid}

\title{Présentation projet TP électronique}
\author{Julia OUZZINE - Pierre-Emmanuel NOVAC}

%\institute{\image{polytechnice.png}{0.05}}
\date{\today}

\begin{document}

\begin{frame}
  \maketitle
\end{frame}

%%\makeatletter
\begin{frame}
\frametitle{Cube de DELs}
\image{Led_cube2.jpg}{0.8}
  \tiny \textit{Source: Mitch Altman — CC-BY-SA — flickr.com}
\end{frame}
%%\makeatother


\begin{frame}
  \frametitle{Pourquoi?}
\end{frame}

\begin{frame}
  \frametitle{Fonctionnalités}
  \begin{itemize}
    \item Contrôle indépendant de 27 DELs
    \item Différentes animations préprogrammées
    \item Choix de l'animation à l'aide d'une application Android (transmission Bluetooth)
    \item Détection de l'orientation et réaction en conséquence
    \item Interaction avec l'environnement: capteur de contact (ou capteur sonore)
  \end{itemize}
\end{frame}

\begin{frame}
  \frametitle{Composants utilisés}
  \begin{itemize}
    \item 1 microcontrôleur: Arduino
    \item 27 DELs bleues
    \item 2 Contrôleur de LED: TI TLC5940
    \item 1 module bluetooth: HC-06
    \item 1 accéléromètre: ADXL335
  \end{itemize}
\end{frame}

\begin{frame}
  \frametitle{Schéma fonctionnel}
  \includesvg{block_diagram}
\end{frame}

\begin{frame}
  \frametitle{Consommation électrique — Alimentation}
  \begin{itemize}
    \item 27 DELs 20mA: $27\times 20 = 540mA$
    \item Arduino: $\approx 50mA$
    \item 2 TLC5940: $2\times 12 = 24mA$
    \item HC-06: $8mA$
    \item ADXL335: $0,35mA$
  \end{itemize}
    $\approx 652mA$\\
      USB: 5V — 500mA max.
\end{frame}


\end{document}





